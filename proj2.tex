\documentclass[a4paper, twocolumn, 11pt]{article}
\usepackage[T1]{fontenc}
\usepackage{graphicx} % Required for inserting images
\usepackage[left=1.4cm, text={18.3cm, 25.2cm}, top=2.3cm,]{geometry}
\usepackage{lmodern}
\usepackage{amsmath}
\usepackage{cancel}
\usepackage{setspace}
\usepackage[utf8]{inputenc}
\usepackage{amsthm}

\begin{document}
\newtheorem{definition}{Definice}

\begin{titlepage}
    \begin{center}
    \Huge \textsc{Vysoké učení technické v Brně} \\
    \vspace{0.5em}
    \huge \textsc{Fakulta informačních technologií} \\
    \vspace{\stretch{0.382}}
    \LARGE Typografie a publikování \,--\, 2. projekt \\
    \vspace{0.6em}
    Sazba dokumentů a matematických výrazů 
    \vspace{\stretch{0.618}}
    \end{center}
    {\Large 2024 \hfill Matyáš Harvánek (xharvam00)}
\end{titlepage}

\newpage
\section*{Úvod}
\par V této úloze si vysázíme titulní stranu a kousek matematického textu, v němž se vyskytují například Definice ~\ref{def1} nebo rovnice ~\eqref{eq2} na straně ~\pageref{eq2}. Pro vytvoření těchto odkazů používáme kombinace příkazů 
\verb|\label|,
\verb|\ref|, 
\verb|\eqref| a 
\verb|\pageref|. Před odkazy patří nezlomitelná mezera. Pro zvýrazňování textu se používají příkazy 
\verb|\verb| a 
\verb|\emph|.
\par Titulní strana je vysázena prostředím 
\verb|titlepage| a nadpis je v optickém středu s využitím 
\emph{přesného} zla\-tého řezu, který byl probrán na přednášce. Dále jsou na titulní straně čtyři různé velikosti písma a mezi dvojicemi řádků textu je použito řádkování se zadanou relativní velikostí 0,5\,em a 0,6\,em\footnote{Použijte správný typ mezery mezi číslem a jednotkou}.

\section{Matematický text}
\par Matematické symboly a výrazy v plynulém textu jsou v prostředí 
\verb|math|. Definice a věty sázíme v prostředí definovaném příkazem 
\verb|\newtheorem| z balíku 
\verb|amsthm|. Tato prostředí obracejí význam 
\verb|\emph|: uvnitř textu sázeného kurzívou se zvýrazňuje písmem v základním řezu. Někdy je vhodné použít konstrukci 
\verb|${}$| nebo 
\verb|\mbox{}|, která zabrání zalomení (matematic\-kého) textu. Pozor také na tvar i sklon řeckých~písmen: srovnejte 
\verb|\epsilon| a 
\verb|\varepsilon|, 
\verb|\Xi| a 
\verb|\varXi|.


\begin{definition}
\label{def1}
\textnormal{Konečný přepisovací stroj} neboli 
\textnormal{Mea\-lyho automat} je definován jako uspořádaná pětice tvaru 
$M = (Q, \varSigma, \varGamma, \delta, q_0)$, kde:


\begin{itemize}
  \item $Q$ je konečná množina \textnormal{stavů},
  \item $\varSigma$ je konečná \textnormal{vstupní abeceda},
  \item $\varGamma$ je konečná \textnormal{výstupní abeceda},
  \item $\delta: Q \times \varSigma \rightarrow Q \times \varGamma$ je totální \textnormal{přechodová funkce},
  \item $q_0 \in Q$ je \textnormal{počáteční stav}.
\end{itemize}


\end{definition}


\subsection{Podsekce s definicí}
\par Pomocí přechodové funkce $\delta$ zavedeme novou funkci~$\delta^*$
    pro překlad vstupních slov $u \in \varSigma^*$ do výstupních slov $w \in \varGamma^*$.

\begin{definition}
    Nechť $M = (Q, \varSigma, \varGamma, \delta, q_0)$ je Mealyho automat. 
    \textnormal{Překládací funkce} $\delta^* : Q \times \varSigma^* \times \varGamma^* \rightarrow \varGamma^*$
    je pro každý stav $q \in Q$, symbol $x \in \varSigma$, slova $u \in \varSigma^*$,
    $w \in \varGamma^*$ definována rekurentním předpisem:  
    \begin{itemize}
    \item $\delta^*(q,\varepsilon,w) = w$  
    \item $\delta^*(q,xu,w) = \delta^*(q',u,wy)$, kde $(q', y) = \delta(q,x)$  
    \end{itemize}  
\end{definition}

\subsection{Rovnice}
    \par Složitější matematické formule sázíme mimo plynulý text  pomocí prostředí \verb|displaymath|. Lze umístit i více výrazů na jeden řádek, ale pak je třeba tyto vhodně oddělit, například pomocí \verb|\quad|, při dostatku místa i~\verb|\qquad|.
\begin{displaymath}
g^{a_n} \notin A^{B^n} \qquad y^1_0 - \sqrt[5]{x + \sqrt[7]{y}} \qquad x > y^2 \geq y^3  
\end{displaymath}
\par Velikost závorek a svislých čar je potřeba přizpůsobit jejich obsahu. Velikost lze stanovit explicitně, anebo pomocí \verb|\left| a \verb|\right|. Kombinační čísla sá\-zejte makrem \verb|\binom|.
\begin{displaymath}
    \left| \bigcup P \right| = \sum\limits_{\emptyset \neq X \subseteq P} (-1)^{|X|-1} \left| \bigcap X \right|
\end{displaymath}
\begin{displaymath}
    F_{n+1} = \binom{n}{0} + \binom{n-1}{1} + \binom{n-2}{2}+ \cdots + \binom{\left\lceil\frac{n}{2}\right\rceil}{\left\lfloor\frac{n}{2}\right\rfloor}
\end{displaymath}


\par V rovnici~\eqref{eq1} jsou tři typy závorek s různou \emph{explicitně} definovanou velikostí. Obě rovnice mají svisle zarovnaná rovnítka. Použijte k tomu vhodné prostředí.

\begin{eqnarray}
    \biggl( \Bigl\{ b \otimes [c_1 \oplus c_2] \circ a \Bigr\}^{\frac{2}{3}} \biggr) & = & \log_zx \label{eq1} \\
    \int_b^a f(x) \mathrm{d}x & = & -\int_b^a f(y)\mathrm{d}y \label{eq2}
\end{eqnarray}


\par \noindent V této větě vidíme, jak se vysází proměnná určující
limitu v běžném textu: $\lim_{m\to\infty}f(m)$. Podobně je to i s dalšími symboly jako 
$\bigcup_{N\in \mathcal{M}}N \textnormal{ či } \sum^m_{ i=1 }x^2_i$. S vynucením méně úsporné sazby příkazem \verb|\limits| budou
vzorce vysázeny v podobě $\lim\limits_{m\to\infty}f(m) \textnormal{ a} \sum\limits^m_{i=1}x^2_i$.

\section{Matice}
\par Pro sázení matic se používá prostředí \verb|array| a závorky
s výškou nastavenou pomocí \verb|\left|, \verb|\right|.
\begin{displaymath}
   D =
   \left|
    \begin{array}{cccc}
    a_{11} & a_{12} & \cdots & a_{1n} \\
    a_{21} & a_{22} & \cdots & a_{2n} \\
    \vdots & \vdots & \ddots & \vdots \\
    a_{m1} & a_{m2} & \cdots & a_{mn}
    \end{array}
    \right|
    =
    \left|
    \begin{array}{cc}
        x & y \\
        t & w
    \end{array}
    \right|
    = xw - yt
\end{displaymath}


\par Prostředí \verb|array| lze úspěšně využít i jinde, například
na pravé straně následující rovnosti.
\begin{displaymath}
    \binom{n}{k} =
    \begin{array}{ll}
        \frac{n!}{k!(n-k)!} & \mathrm{pro} \ 0 \leq k \leq n \\
        0 & \mathrm{jinak}
    \end{array}
\end{displaymath}
    
\end{document}
